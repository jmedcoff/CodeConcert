\documentclass{beamer}
\usepackage{amsmath}
\usepackage{amssymb}
\usepackage{textpos} % package for the positioning
\usepackage{graphicx} % Allows including images
\usepackage{tikz} % for those who hate themselves
\usetikzlibrary{calc}
\usepackage{mwe}
\usetikzlibrary{positioning}
\usepackage{listings}
\usepackage{fontenc}

\usetheme{Madrid}


\title[Code Concert]{Code Concert: Writing live music with TidalCycles}
\author{Jason Medcoff}
\date{September 22 2018}

\begin{document}
	\begin{frame}
		\frametitle{\space}
		\titlepage
	\end{frame}

	\begin{frame}
		\frametitle{Introduction}
		A little bit about me.
		
		I'm a programmer, a musician, and a mathematics enthusiast.
		
		I'm very excited today to give a talk involving all three.
		
		If this excites you too, you're in the right place.
	\end{frame}

	\begin{frame}
		\frametitle{Introduction}
		What is this talk about?
		
		\begin{itemize}
			\item TidalCycles, a pattern based language
			\item Haskell, the programming language it's built on
			\item How to put them together to have a lot of fun
		\end{itemize}
	
	Functional programming is really awesome, therefore Haskell is awesome, though neither are the main focus here. We'll talk about it enough to understand what's going on.
	\end{frame}

	\begin{frame}
		\frametitle{Lists and You: the joy of FP}
		Lists are the quintessential data type. You can't do much without them.
		
		\begin{itemize}
			\item plain old arrays
			\item linked lists
			\item \texttt{java.util.ArrayList}
			\item \texttt{'()} or \texttt{nil}
			\item etc.
		\end{itemize}
	\end{frame}

	\begin{frame}
		\frametitle{Lists and You}
		FP is the best way to work with lists, because no matter the language, abstract lists are recursively defined.
		
		\begin{center}
			\texttt{<List> = <[]> | <Object><List>}
		\end{center}
	
		By the way, this is basically all of lisp.
		\begin{center}
			\texttt{(function first-arg second-arg etc...)}
		\end{center}	
	\end{frame}

	\begin{frame}
		\frametitle{Lists and You}
		We can build lists by hand.
		
			\begin{itemize}
				\item \texttt{1 : []} gives \texttt{[1]}
				\item \texttt{1 : [2, 3]} gives \texttt{[1, 2, 3]}
				\item \texttt{[1, 2] : [[3]]} gives \texttt{[[1, 2], [3]]}
				\item \texttt{'h' : 'e' : 'l' : 'l' : 'o' : []} gives \texttt{"hello"}
			\end{itemize}
	\end{frame}

	\begin{frame}
		\frametitle{Lists and You}
		Some takeaways:
		\begin{itemize}
			\item Haskell is strongly typed, and thus wants homogeneous lists
			\item Haskell treats strings as lists of characters
		\end{itemize}
	\end{frame}

	\begin{frame}[fragile]
		\frametitle{Enter TidalCycles: the Pattern}
		Let's start with a notion of tempo.
		
		Tidal measures cycles per second (cps), but we can use some \textit{DIMENSIONAL ANALYSIS}$^{TM}$ to put this in more familiar terms.
		
		\begin{lstlisting}[language=Haskell]
		cps = (120/60/4)
		\end{lstlisting}
		
		where 120 is beats per minute (bpm), 60 is seconds per minute, and 4 is beats per cycle. We are not restricted to this division.
	\end{frame}

	\begin{frame}
		\frametitle{The Pattern}
		What is a ``list" in music?
		\begin{itemize}
			\item a lick
			\item a riff
			\item a melody
			\item a beat
			\item a bassline
			\item etc.
		\end{itemize}
	\end{frame}

	\begin{frame}
		\frametitle{An example: the L I C C}
		
	\end{frame}

	\begin{frame}[fragile]
		\frametitle{An example: the L I C C}
		\begin{lstlisting}[language=Haskell,showstringspaces=false]
		d1 $ slow 4 $ up 
		"[2 4 5 7 4 ~ 0 2] ~" # s "arpy"
		\end{lstlisting}
		\begin{itemize}
			\item sample references
			\item rests
			\item sequences
			\item functions, of course
		\end{itemize}
	\end{frame}

	\begin{frame}
		\frametitle{Extreme hand waving}
		To summarize:
		\begin{center}
			\texttt{<Sample-ref> = <Sample-name>":"<Sample-index>}
			
			\texttt{<Note> = <\textasciitilde> | <Sample-ref>}
			
			\texttt{<Notes> = < > | <Note><Notes>}
			
			\texttt{<Pattern> = <Notes> | <Notes><Pattern><Notes>}
			
		\end{center}
	\end{frame}

	\begin{frame}
		\frametitle{The TidalCycles Architecture}
		Audio applications that aren't workstations tend to have a simple and familiar paradigm.
		\begin{itemize}
			\item A client, interfaced by an editor
			\item A server, listening for messages
		\end{itemize}
	\end{frame}

	\begin{frame}
		\frametitle{The TidalCycles Architecture}
		Here's how Tidal does it.
		
		The client is a Haskell interactive session running TidalCycles.
		
		The server is a sampler running on SuperCollider.
	\end{frame}

	\begin{frame}[fragile]
		\frametitle{Baby Steps}
		Let's make some noise.
		
		\begin{lstlisting}[language=Haskell]
		d1 (pattern-effect (... 
		(sound "some-pattern" 
		# synth-effects)))
		\end{lstlisting}
		
		Make it concise:
		\begin{lstlisting}[language=Haskell]
		d1 $ pattern-effect $ 
		sound "some-pattern" # synth-effects
		\end{lstlisting}
	\end{frame}

	\begin{frame}
		\frametitle{Examples}
	\end{frame}

	\begin{frame}[fragile]
		\frametitle{Examples}
		What can programs do that human performers or sequencing software can't easily do?
		
		\begin{lstlisting}[language=Haskell]
		d1 $ s "rad1*8" # n $ run 8
		\end{lstlisting}
		
		Pay close attention to the \texttt{n} function. Then try \texttt{irand}.
	\end{frame}

	\begin{frame}
		\frametitle{Samples}
		Where do the samples come from?
		
		I like to prepare my own, and combine them in interesting ways.
	\end{frame}

	\begin{frame}
		\frametitle{Where to go from here}
		Interested? Check this out.
		\begin{itemize}
			\item tidalcycles.org 
			\item TOPLAP (toplap.org)
			\item bit.ly/codeconcert
		\end{itemize}
		
	\end{frame}











\end{document}